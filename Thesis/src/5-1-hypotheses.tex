\subsection{Hypotheses}
	We will discuss the results of our hypotheses tests in the same order as we introduced them.

	\paragraph{Hypothesis Ai}
	\lqq A \(10\%\) correct assistance system increases the correctness of annotations compared to annotations done without the assistance.\rqq

	Since \(p > \alpha\), this result is \textbf{not significant}; the $H0$ is not rejected.
	Consequently the assistance system did (statistically) not support the annotation task in terms of correctness, when it only made 10\% correct (and 90\% wrong) suggestions. The mean difference between blocks with assistance and blocks without is $-0.62\%$. Since this value is negative, the assistance system with 10\% correct suggestions did (on a descriptive level) impair the correctness that participants could achieve using this system.

	\paragraph{Hypothesis Aii}
	\lqq A \(50\%\) correct assistance system increases the correctness of annotations compared to annotations done without the assistance.\rqq

	\(\Rightarrow p < \alpha\); this result \textbf{is significant}; the $H0$ is rejected.
	Here, the assistance system did (statistically) support the annotation task in terms of correctness, when it made 50\% correct (and 50\% wrong) suggestions. On average, the annotations were done about 3.7\% more correctly with the assistance present than without its presence. This is an increase of the baseline from 83.93\% to 87,63\%.

	\paragraph{Hypothesis Aiii}
	\lqq A \(90\%\) correct assistance system increases the correctness of annotations compared to annotations done without the assistance.\rqq

	\(\Rightarrow p < \alpha\); this result \textbf{is significant}; the $H0$ is rejected.
	Consequently the assistance system did (statistically) support the annotation task in terms of correctness, when it made 90\% correct (and 10\% wrong) suggestions. On average, the annotations were done about 6.1\% more correctly with the assistance present than without its presence. This is an increase of the baseline from 83.93\% to 90,03\%. This result is not too surprising since the 50\% assistance produced already significant results. It is remarkable that such a system helps to reach over 90\% of annotation correctness on average.



	\paragraph{Hypothesis Bi}
	\lqq A \(10\%\) correct assistance system decreases the average time needed for one annotation compared to annotations done without the assistance.\rqq

	Since \(p > \alpha\), this result is \textbf{not significant}; the $H0$ is not rejected.
	Consequently the assistance system did (statistically) not support the annotation task in terms of tempo, when it made only 10\% correct (and 90\% wrong) suggestions. The mean difference between blocks with assistance and blocks without is 0.51 seconds. Since this value is greater than zero, the assistance system with 10\% correct suggestions did (on a descriptive level) cause a longer time needed to make a single correct annotation. The baseline there is 8.19 seconds without an assistance.

	\paragraph{Hypothesis Bii}
	\lqq A \(50\%\) correct assistance system decreases the average time needed for one annotation compared to annotations done without the assistance.\rqq

	\(\Rightarrow p < \alpha\); this result \textbf{is significant}; the $H0$ is rejected.
	Consequently the assistance system did (statistically) support the annotation task in terms of tempo, when it made 50\% correct (and 50\% wrong) suggestions. On average, the annotations were done about 1.7s (per annotation) faster with this assistance than without it. This is a reduction of the baseline from 8.19 seconds to 6,49 seconds per correct annotation.

	\paragraph{Hypothesis Biii}
	\lqq A \(90\%\) correct assistance system decreases the average time needed for one annotation compared to annotations done without the assistance.\rqq

	\(\Rightarrow p < \alpha\); this result \textbf{is significant}; the $H0$ is rejected.
	Consequently the assistance system did (statistically) support the annotation task in terms of tempo, when it made 90\% correct (and 10\% wrong) suggestions. On average, the annotations were done about 1.9s (per annotation) faster when this assistance was present than without it. This is a reduction of the baseline from 8.19 seconds to 6,29 seconds per correct annotation, or a decrease of more than 23\% of the time.



	\paragraph{Hypothesis Ci}
	\lqq A \(10\%\) correct assistance system decreases the miss rate per block compared to annotating a block of text done without the assistance.\rqq

	Since \(p > \alpha\), this result is \textbf{not significant}; the $H0$ is not rejected.
	Consequently the assistance system did (statistically) not support the annotation task in terms of reducing the rate of missed annotations, when it made only 10\% correct (and 90\% wrong) suggestions. The mean difference between blocks with assistance and blocks without is $-1.55\%$. Descriptively this is a step in the right direction, but since it is not significant, such an assistance would not make an impact. The baseline of the miss rate is 7.69\% of missed annotations without an assistance.

	\paragraph{Hypothesis Cii}
	\lqq A \(50\%\) correct assistance system decreases the miss rate per block compared to annotating a block of text done without the assistance.\rqq

	\(\Rightarrow p < \alpha\); this result \textbf{is significant}; the $H0$ is rejected.
	Consequently the assistance system did (statistically) support the annotation task in terms of reducing the rate of missed annotations, when it made 50\% correct (and 50\% wrong) suggestions. On average, the rate of missed annotations is decreased of about 2.8\% if the assistance was present. This is a reduction of the baseline from 7.69\% to 4,89\% of overall missed annotations. This is an improvement of more than 36\%.

	\paragraph{Hypothesis Ciii}
	\lqq A \(90\%\) correct assistance system decreases the miss rate per block compared to annotating a block of text done without the assistance.\rqq

	\(\Rightarrow p < \alpha\); this result \textbf{is significant}; the $H0$ is rejected.
	Consequently the assistance system did (statistically) support the annotation task in terms of reducing the rate of missed annotations, when it made 90\% correct (and 10\% wrong) suggestions. On average, the rate of missed annotations is decreased of about 3.94\% if the assistance was present. This is a reduction of the baseline from 7.69\% to 3,75\% of overall missed annotations. This is an improvement of more than 51\%.



	\paragraph{Hypothesis Di}
	\lqq The \(50\%\) correct assistance system will increase the influence it has on correctness compared to the \(10\%\) correct assistance system.\rqq

	Since \(p > \alpha\), this result is \textbf{not significant}; the $H0$ is not rejected.
	Consequently the assistance system in the levels 10\% and 50\% are, statistically, not different regarding their influence on the correctness of the participants' annotations.

	\paragraph{Hypothesis Dii}
	\lqq The \(90\%\) correct assistance system will increase the influence it has on correctness compared to the \(50\%\) correct assistance system.\rqq

	Since \(p > \alpha\), this result is \textbf{not significant}; the $H0$ is not rejected.
	Consequently the assistance system in the levels 50\% and 90\% are, statistically, not different regarding their influence on the correctness of the participants' annotations.

	\paragraph{Hypothesis Ei}
	\lqq The \(50\%\) correct assistance system will increase the influence it has on tempo compared to the \(10\%\) correct assistance system.\rqq

	Since \(p > \alpha\), this result is \textbf{not significant}; the $H0$ is not rejected.
	Consequently the assistance system in the levels 10\% and 50\% are, statistically, not different regarding their influence on the tempo of the participants' annotations.

	\paragraph{Hypothesis Eii}
	\lqq The \(90\%\) correct assistance system will increase the influence it has on tempo compared to the \(50\%\) correct assistance system.\rqq

	Since \(p > \alpha\), this result is \textbf{not significant}; the $H0$ is not rejected.
	Consequently the assistance system in the levels 50\% and 90\% are, statistically, not different regarding their influence on the tempo of the participants' annotations.

	\paragraph{Hypothesis Fi and Fii}
	\lqq The \(50\%\) correct assistance system will increase the influence it has on the miss rate compared to the \(10\%\) correct assistance system.\rqq

	Using the \ac{ANOVA} we did not find a significant main effect of the level of the assistance system. Hence we did no further investigation and did not compute contrasts between the the levels of the assistance. The Fi related $H0$ as well as the Fii related $H0$ are not rejected.

	Interestingly, we found a significant main effect of the block, expressing the point of time of measurement (the first half or the second half). This shows that the participants varied in their success in identifying all annotations in the first and in the second half of the study. A learning effect or a tiredness effect could explain this observation. Since we had no hypothesis regarding these effects, we did not analyze this any further.


	\paragraph{Summary}
	To summarize these results, Table~\ref{tab:DiscussionHypothesesWithResults} shows which of the hypotheses were significant. Compare Table~\ref{tab:Hypotheses}.

	\begin{table}\centering
		\caption{List of hypotheses, highlighting significant results.}
		\begin{tabular}{clll}
			\toprule
			& correctness & tempo & misses \\
			\midrule
			$10\%$ correct Assistance & increase \XSolidBrush & decrease \XSolidBrush & decrease \XSolidBrush \\
			$50\%$ correct Assistance & \cellcolor{HighlightGreen}increase \Checkmark & \cellcolor{HighlightGreen}decrease \Checkmark & \cellcolor{HighlightGreen}decrease \Checkmark \\
			$90\%$ correct Assistance & \cellcolor{HighlightGreen}increase \Checkmark & \cellcolor{HighlightGreen}decrease \Checkmark & \cellcolor{HighlightGreen}decrease \Checkmark \\
			$10\%$ vs. $50\%$ & increase \XSolidBrush & decrease \XSolidBrush & decrease \XSolidBrush \\
			$50\%$ vs. $90\%$ & increase \XSolidBrush & decrease \XSolidBrush & decrease \XSolidBrush \\
			\bottomrule
		\end{tabular}
		\label{tab:DiscussionHypothesesWithResults}
	\end{table}



	\subsubsection{Questionnaire Evaluation: Perceived Workload}
		This analysis is purely explorative. We established no hypotheses regarding the questionnaire.

		\paragraph{assistance level 10\%}
		Since \(p > \alpha\), this result is \textbf{not significant}.
		Consequently, the assistance system did not (statistically) support the annotation task in terms of reducing the perceived workload of the participants, when it made only 10\% correct (and 90\% wrong) suggestions. This is not surprising when we put this in relation to the results we obtained from our hypotheses tests regarding the 10\% assistance. None of these tests were significant and we concluded that such a system does not support the task of annotation. Therefore it seems consistent that this system does not significantly influence the perceived workload of the participants.

		\paragraph{assistance level 50\%}
		Since \(p > \alpha\), this result is \textbf{not significant}.\\
		Consequently the assistance system did not (statistically) support the annotation task in terms of reducing the perceived workload of the participants, when it made only 50\% correct (and 50\% wrong) suggestions. We would expect a significant impact on the perceived workload based on what we have analyzed using the same system so far -- this is the first test with the 50\% correct assistance that is not statistically significant. Thus the influence this system has is with an average of $-0.32$ points at a 7 point Likert scale still very marginal.
		\vspace{0.5cm}

		\paragraph{assistance level 90\%}
		\(\Rightarrow p < \alpha = 0.05\); this result \textbf{is significant}.\\
		Consequently, the assistance system did (statistically) support the annotation task in terms of reducing the perceived workload of the participants, when it made 90\% correct (and 10\% wrong) suggestions. On average, the perceived workload of the participants decreased about 0.7 points (at a 7 point Likert scale) when this assistance was present. The assistance system with 90\% correct annotations does influence the perceived workload significantly. This seems plausible as the task of overseeing very correct pre-annotations is not very demanding but rather tedious.

		\paragraph{\ac{ANOVA}}
		The results of the \ac{ANOVA} testify a significant main effect of the level of the assistance. As we tested beforehand using the t-tests, we found, that the 90\% correct assistance has a significant impact on the perceived workload of the participants. Besides this, there is no significant main effect of the block (the point of time of measurement) and no significant interaction effect of the level and the block.



	\subsubsection{Questionnaire Evaluation: Perceived Monotony}
		\paragraph{assistance level 10\%}
		Since \(p > \alpha\), this result is \textbf{not significant}.\\
		Consequently, the assistance system did not (statistically) support the annotation task in terms of reducing the perceived monotony of the participants when it made only 10\% correct (and 90\% wrong) suggestions. Similar to the result of the 10\% assistance regarding the perceived workload, this is not surprising. The suggestions of this assistance system have too little impact to assist the annotation task. On a descriptive level, the average difference between blocks with and without assistance is positive and therefore impairs the perceived monotony.

		\paragraph{assistance level 50\%}
		Since \(p > \alpha\), this result is \textbf{not significant}.\\
		Consequently the assistance system did (statistically) not support the annotation task in terms of reducing the perceived monotony of the participants when it made only 50\% correct (and 50\% wrong) suggestions. It is notable that, descriptively, the average difference between blocks with and without assistance is reduced by about 0.34 points of the Likert scale. Unfortunately this is not a statistically significant result.

		\paragraph{assistance level 90\%}
		Since \(p > \alpha\), this result is \textbf{not significant}.\\
		Consequently, the assistance system did not (statistically) support the annotation task in terms of reducing the perceived monotony of the participants when it made 90\% correct (and 10\% wrong) suggestions. Similar to the results of the 10\% assistance, this system seems -- on a descriptive level -- to impair the perceived monotony. A very bad and a very good assistance seem to bore the participant. One because it is frustratingly incorrect, the other because the assistance seems to not leave anything to annotate over.

		Because none of the t-tests showed any significant results we did not investigate the data regarding the perceived monotony any further. We conclude, that all the different levels of the system did not influence the participants in this dimension; the task of annotating text seems to be truly monotonous.

	\subsubsection{Questionnaire Evaluation: Perceived Reliability of the Assistance}
		\(\Rightarrow\) The \(p\)-value of the level of the assistance system is less than our significance level of \(\alpha = 0.05\), this result \textbf{is significant}; we found a main effect of the level of the assistance. Besides this there is no significant main effect of the block (the point of time of measurement) and no significant interaction effect of the level and the block. No main effect of the block means that the perceived reliability of the assistance did not change over time. To be exact: It did not change between the first and the second halves. The main effect of the level of the assistance is plausible too -- the better the assistance (regarding its level), the more reliable the participants perceived it.

		On the 7 point Liker scale, the average score of the 10\% assistance was 2.68; the 50\% assistance was rated with 3.89 on average and the 90\% assistance was rated with a score of 4.93 on average. On the one hand, the 10\% assistance was rated quiet fair if we take into account that it had a mistake rate of 90\%. On the other hand, the 90\% correct assistance was rated disproportionately low, given the fact that it made 90\% correct suggestions.

	\subsubsection{Questionnaire Evaluation: Perceived Correctness of the Assistance}
		\(\Rightarrow\) The \(p\)-value of the level of the assistance system is less than our significance level of \(\alpha = 0.05\), this result \textbf{is significant}; we found a main effect of the level of the assistance. Besides this, there is no significant main effect of the block (the point of time of measurement) and no significant interaction effect of the level and the block. These results are comparable to the evaluation of the perceived reliability: The perceived correctness did not change over time (no main effect of the block) and its correctness was rated proportionally to its actual correctness (main effect of the level).

		It is remarkable that the participants again overestimated the poorest assistance but underestimated the best one. They did rate the 50\% and the 90\% correct assistance quite well (they missed their actual correctness by 7.32\% and 10,25\%), but overestimated the 10\% correct assistance by almost 26\%. We assume that the 90\% mistakes that the poorest assistance made were not perceived as bad as they actually are. Since it made different mistakes (see Section~\ref{sec:simulationOfTheAssistanceSystem}), we believe that they are not equally hard to improve.\footnote{In fact, we encourage the interested reader to analyze our data set if the different mistakes are differently hard to find and to improve. See Section~\ref{sec:discussFurtherResearch}.} We assume for example that a correct labeled chunk with just a little offset in its span is fairly easy to improve -- for example easier than a missed annotation. Since 17.55\% (of the 90\% mistakes that the poorest assistance system made, see Section~\ref{sec:simulationOfTheAssistanceSystem}) are annotations with just a wrong span, it seems plausible that such mistakes were not perceived to be as wrong as they actually are.
