\subsection{Impact of the Assistance System on the Three Performance Parameters}
	\label{sec:resultsHypAC}
	This section presents our analysis results of the t-tests we conducted for the hypotheses groups A, B and C.
	All tests in this section used a significance level of \(\alpha = 0.05\).

	\subsubsection{Impact of the Assistance on Correctness}
		Figure~\ref{fig:AnalysisCorrectness} shows a comparison of the three different assistance levels regarding the correctness of the participants' annotations. The level with 10\% correct pre-annotations was not significantly greater than zero, whereas the level 50\% and the level 90\% assistance revealed significant results.

		\singleFig{4/analysis_correctness.png}{All three levels of assistance regarding the performance dimension correctness. The height of each bar is the mean value of the condition, the whiskers depict the standard error of the mean in both directions.}{AnalysisCorrectness}

		\paragraph{Hypothesis Ai}
		\lqq A \(10\%\) correct assistance system increases the correctness of annotations compared to annotations done without the assistance.\rqq

		There was no significant difference, \(t(21) = -0.521, p = 0.608\ (M = -0.62, SD = 5.601)\), so the null hypothesis is not rejected.

		\paragraph{Hypothesis Aii}
		\lqq A \(50\%\) correct assistance system increases the correctness of annotations compared to annotations done without the assistance.\rqq

		There was a significant difference, \(t(21) = 2.123, p = 0.046\ (M = 3.67, SD = 8.107)\), so the null hypothesis is rejected at the given level of significance.

		\paragraph{Hypothesis Aiii}
		\lqq A \(90\%\) correct assistance system increases the correctness of annotations compared to annotations done without the assistance.\rqq

		There was a significant difference, \(t(21) = 4.367, p < 0.001\ (M = 6.05, SD = 6.502)\), so the null hypothesis is rejected at the given level of significance.



	\subsubsection{Impact of the Assistance on Tempo}
		Figure~\ref{fig:AnalysisTempo} shows a comparison of the three different assistance levels regarding the tempo of the participants. The level with 10\% correct pre-annotations was not significantly less than zero, whereas the level 50\% and the level 90\% assistance revealed significant results.

		\singleFig{4/analysis_tempo.png}{All three levels of assistance regarding the performance dimension correctness. The height of each bar is the mean value of the condition, the whiskers depict the standard error of the mean in both directions.}{AnalysisTempo}

		\paragraph{Hypothesis Bi}
		\lqq A \(10\%\) correct assistance system decreases the average time needed for one annotation compared to annotations done without the assistance.\rqq

		There was no significant difference, \(t(21) = 0.762, p = 0.454\ (M = 0.51, SD = 3.161)\), so the null hypothesis is not rejected.

		\paragraph{Hypothesis Bii}
		\lqq A \(50\%\) correct assistance system decreases the average time needed for one annotation, compared to annotations done without the assistance.\rqq

		There was a significant difference, \(t(21) = -2.151, p = 0.043\ (M = -1.65, SD = 3.608)\), so the null hypothesis is rejected at the given level of significance.

		\paragraph{Hypothesis Biii}
		\lqq A \(90\%\) correct assistance system decreases the average time needed for one annotation compared to annotations done without the assistance.\rqq

		There was a significant difference, \(t(21) = -6.416, p < 0.001\ (M = -1.93, SD = 1.412)\), so the null hypothesis is rejected at the given level of significance.



	\subsubsection{Impact of the Assistance on Misses}
		Figure~\ref{fig:AnalysisMisses} shows a comparison of the three different assistance levels regarding the miss rate of the participants. The level with 10\% correct pre-annotations was not significantly less than zero, whereas the level 50\% and the level 90\% assistance revealed significant results.

		\singleFig{4/analysis_misses.png}{All three levels of assistance regarding the performance dimension misses. The height of each bar is the mean value of the condition, the whiskers depict the standard error of the mean in both directions.}{AnalysisMisses}

		\paragraph{Hypothesis Ci}
		\lqq A \(10\%\) correct assistance system decreases the miss rate per block compared to annotating a block of text done without the assistance.\rqq

		There was no significant difference, \(t(21) = -1.176, p = 0.252\ (M = -1.55, SD = 6.176)\), so the null hypothesis is not rejected.

		\paragraph{Hypothesis Cii}
		\lqq A \(50\%\) correct assistance system decreases the miss rate per block compared to annotating a block of text done without the assistance.\rqq

		There was a significant difference, \(t(21) = -2.383, p = 0.027\ (M = -2.80, SD = 5.508)\), so the null hypothesis is rejected at the given level of significance.

		\paragraph{Hypothesis Ciii}
		\lqq A \(90\%\) correct assistance system decreases the miss rate per block compared to annotating a block of text done without the assistance.\rqq

		There was a significant difference, \(t(21) = -3.591, p = 0.002\ (M = -3.94, SD = 5.147)\), so the null hypothesis is rejected at the given level of significance.
